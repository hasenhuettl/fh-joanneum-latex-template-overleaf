\chapter{Ausgewählte Formatierungsbeispiele}

\section{Ein nummerierter Abschnitt}

\subsection{Ein nummerierter Unterabschnitt}

\subsubsection{Ein nummerierter Unterunterabschnitt}

Welche Überschriftenebenen nummeriert werden wird mittels \texttt{secnumdepth} festgelegt.

\subsubsection{Ein nummerierter Unterunterabschnitt}

Jede Gliederungsebene muss zumindest zweimal vorkommen. Dies ist \textbf{fett}, \textit{kursiv}, \underline{unterstrichen} oder \texttt{Schreibmaschine}.

\section{Mathematik}

Im Ortsraum ergibt sich der eindimensionale Hertzvektor $\Phi_A(\vec{r})$ für das gesamte Teilchengitter zu

\begin{equation}
\Phi_A(\vec{r})=\sum_m\sum_n\exp\left( i\left[ k_x^imd+k_y^ind\right] \right) \Phi\left( \vec{r}-\vec{r}_{mn}\right)
\end{equation}

\section{Code}

Es gibt mehrere Möglichkeiten Code professionell darzustellen; hier mittels des \LaTeX-Paketes \texttt{Listings}. Die Gestaltungsmöglichkeiten sind umfangreich.

\lstdefinestyle{customerlang}{
	belowcaptionskip=1\baselineskip,
	breaklines=true,
	frame=L,
	xleftmargin=\parindent,
	language=erlang,
	showstringspaces=false,
	basicstyle=\footnotesize\ttfamily,
	keywordstyle=\bfseries\color{green!40!black},
	commentstyle=\itshape\color{purple!40!black},
	identifierstyle=\color{blue},
	stringstyle=\color{orange},
}

\lstset{language=erlang, style=customerlang}

\begin{lstlisting}[caption=Das übliche 'Hello World' Spektakel -- diesmal in Erlang.]
-module(hello).
-export([start/0]).

start() ->
	io:format("Hello world~n").
\end{lstlisting}

\section{Literaturverweise}

Keine Masterarbeit ohne Quellenangaben \ldots. 

Die Referenzen werden in eine \textsc{Bib}\TeX Datei ausgelagert. Diese kann mit einem beliebigen Texteditor, oder mit spezifischen Editoren erstellt werden (Literaturverwaltung).

Eine Referenz auf einen Artikel in einem Journal~\cite{Brand:1983vx}, auf ein Buch~\cite{Weske:2012ul}, auf ein Buchkapitel in einem Herausgeberband~\cite{singer2015}, sowie auf einen Beitrag in einem Konferenzband~\cite{Singer:2016hl} (Proceedings).

\section{Querverweise}

Ein Verweis auf eine Abbildung ist im Abschnitt~\ref{AnotherSection} und ein Verweis auf eine Tabelle ist im Abschnitt~\ref{AnotherSection} zu finden.

